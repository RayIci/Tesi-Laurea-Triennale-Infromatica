\section{Studio dei metodi e delle applicazioni}
In tale capitolo verranno spiegati i metodi, le applicazioni ed i concetti 
generali studiati durante la prima fase del tirocinio, accostando ad ogni 
di essi una relativa implementazione e/o utilizzo.

\paragraph{Cos'è una serie temporale} In statistica descrittiva, una serie 
storica (o temporale) si definisce come un insieme di variabili 
casuali ordinate rispetto al tempo, ed esprime la dinamica di un certo 
fenomeno nel tempo. Le serie storiche vengono studiate sia per 
interpretare un fenomeno, individuando componenti di trend, di ciclicità, 
di stagionalità e/o di accidentalità, sia per prevedere il suo 
andamento futuro \cite{wiki:serie_storica}.
In altre parole una serie storica (o temporale) è un'insieme/serie di dati
capionati ed indicizzati nel tempo ad intervalli regolari come ore, giorni 
o anni.\\
In termini più matematici indichiamo con $Y$