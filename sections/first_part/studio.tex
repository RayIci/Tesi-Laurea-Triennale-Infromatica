\section{Studio dei metodi e delle applicazioni}
In tale capitolo verranno spiegati i metodi, le applicazioni ed i concetti 
generali studiati durante la prima fase del tirocinio, accostando ad ogni 
di essi una relativa implementazione e/o utilizzo.\\
In primo luogo si troverà una definizione di serie temporale, seguita da
una spiegazione dei metodi per manipolare i dati inerenti a serie temporali 
su python, le componenti principali di una serie temporale e la loro visualizzazione
ed infine metodi per l'analisi di essi.\\
\\
Molti degli esempi forniti in questa sezione fanno riferimento a dataset disponibili
al sito ``UCI Machine Learning Repository''~\cite{dua:2019} più precisamente, come
esempio, è stato utilizzato un dataset relativo alla qualità dell'aria della città di
Beijing~\cite{dua:air_quality} (Pechino).
 
\subsection{Cos'è una serie temporale}
In statistica descrittiva, una serie storica (o temporale) si definisce come un insieme di variabili 
casuali ordinate rispetto al tempo, ed esprime la dinamica di un certo 
fenomeno nel tempo. Le serie storiche vengono studiate sia per 
interpretare un fenomeno, individuando componenti di trend, di ciclicità, 
di stagionalità e/o di accidentalità, sia per prevedere il suo 
andamento futuro~\cite{wiki:serie_storica}.
In altre parole una serie storica (o temporale) è un'insieme/serie di dati
capionati ed indicizzati nel tempo ad intervalli regolari come ore, giorni 
o anni.\\
\\
In termini più matematici: indichiamo con $\bm{Y}$ il fenomeno (ad esempio 
il prezzo della benzina dall'anno $1970$ all'anno $2010$) ed indichiamo con
$\bm{Y_t}$ un'ossevazione al tempo $\bm{t}$, con $\bm{t}$ un numero intero
compreso tra $\bm{1}$ a $\bm{T}$, dove $\bm{T}$ è il numero totale degli intervalli o 
periodi. Una serie temporale viene espressa in questa maniera 
$\bm{Y_t} = \left\{  \bm{Y_1}, \bm{Y_2}, \dots , \bm{Y_T}  \right\}$.

\begin{esempio} [\textit{Prezzo della benzina}]
    Se consideriamo come fenomeno $\bm{Y}$ il prezzo della benzina dal $1970$ al $2010$
    avremmo come numero totale di osservazioni (o numero totale di periodi) 
    $\bm{T} = \bm{40}$ dove:
    \begin{itemize}
        \setlength\itemsep{-0.5em}
        \item $\bm{Y_1}$: prezzo della benzina all'anno $1970$
        \item $\bm{Y_2}$: prezzo della benzina all'anno $1971$
        \item $\bm{Y_T} = \bm{Y_{40}}$: prezzo della benzina all'anno $2010$
    \end{itemize}

\end{esempio}


\subsection{Metodi per la gestione dei dati}
In questo sottocapitolo verranno spiegati i metodi studiati ed utilizzati per la gestione dei
dati in python, più nello specifico come caricare un dataset, una possibile rinomina delle
colonne ralative alle serie per una maggiore comprensione, individuazione dei valori
nulli ed infine una sezione relativa al filtraggio.

\paragraph{Cosa viene inteso per dataset} Per dataset si intende un insieme di
serie (nella nostra applicazione temporali) relative ad un'unica applicazione.

\begin{esempio}[\textit{Qualità dell'aria}]
    Consideriamo, per esempio, come applicazione le misurazioni di diversi parametri
    relativi alla qualità dell'aria di Genova: livello di $\mathsf{CO_2}$,
    livello di $\mathsf{SO_2}$, livello di $\mathsf{NO_2}$ e temperatura in \textdegree$\mathsf{C}$.
    In un dataset possiamo considerare ogni parametro come una serie temporale diversa ma indicizzata
    nel tempo in ugual maniera, quindi se queste misurazioni avvengono ogni ora avemo
    per ogni istante di tempo $t$ le misurazioni per ogni parametro in quell'istante.
    \begin{itemize}
        \setlength\itemsep{-0.5em}
        \item $Y_1$: livello di $\mathsf{CO_2}$, $\mathsf{SO_2}$, $\mathsf{NO_2}$ e temperatura in \textdegree$\mathsf{C}$ all'istante $1$.
        \item $Y_2$: livello di $\mathsf{CO_2}$, $\mathsf{SO_2}$, $\mathsf{NO_2}$ e temperatura in \textdegree$\mathsf{C}$ all'istante $2$.
        \item \dots
        \item $Y_T$: livello di $\mathsf{CO_2}$, $\mathsf{SO_2}$, $\mathsf{NO_2}$ e temperatura in \textdegree$\mathsf{C}$ all'istante $T$.
    \end{itemize}
    Ogni parametro in un dataset viene rappresentato come una colonna.
\end{esempio}

\subsubsection{Pacchetti utilizzati}
Per effettuare tutte le manovre relative all'elaborazione e manipolazione dei dati sono state
utilizzate funzionalità fornite da pacchetti python come pandas e numpy, essi semplificano
la scrittura di codice e velocizzano il tempo di sviluppo organizzando in maniera ottimale
i dati.

\subsubsection{Caricamento di un dataset}





\subsubsection{Rinomina delle colonne relative alle serie}
\subsubsection{Individuazione dei valori nulli e possibili soluzioni}
\subsubsection{Filtraggio}


