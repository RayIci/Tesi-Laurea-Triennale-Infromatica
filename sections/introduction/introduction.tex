\section{Introduzione}
In tale capitolo introduttivo verranno spiegati gli obbiettivi e la suddivisione del tirocinio
in merito al tempo disponibile, cercando di spiegare, in maniera sintetica, come è stato
svolto ed organizzato.

\subsection{Premessa}
Lo scopo di questo elaborato è quello di descrivere l'esperienza di tirocinio svoltasi presso
l'Università Degli Studi di Genova con durata complessiva di 300 ore con inizio 25 novembre 2022
e fine 27 febbraio 2023. Il tirocinio è stato svolto per la maggior parte del tempo da remoto
con incontri, volti all'andamento di esso, mediante l'utilizzo di Teams (piattaforma sviluppata da Microsoft)
che in presenza presso il dipartimento. 

\subsection{Obbiettivi}
L'obbiettivo di questo tirocinio è stato quello di provare a individuare, se esistenti, uno o più metodi, 
relativi all'analisi di serie temporali, che permettano l'analisi di una serie di 
dati relative a soggeti sani e petologici con lo scopo di analizzare il cammino.
\\
I dati utilizzati sono stati analizzati precedentemente da un gruppo di ricerca del dipartimento
con tecniche differenti da quelle utilizzate durante il tirocinio
quindi in caso di un metodo sicuro e soddisfacente all'analisi, i risultati ottenuti sarebbero
serviti al gruppo di ricerca come un'ulteriore conferma delle analisi da loro eseguite.

\subsection{Suddivisione del lavoro}
Per poter garantire la conoscienza necessaria allo sviluppo di un metodo relativo allo scopo
del tirocinio, il lavoro è stato principalmente suddiviso in due fasi: studio dei metodi
relativi all'analisi di serie temporali ed effettivo sviluppo, e ricerca, di un metodo per l'analisi
del problema posto.\\
\\
Nella prima fase sono stati studiati i metodi ed applicazioni di tecniche volte allo studio
di serie temporali, sia da un lato pratico che da un lato teorico. Per quanto riguarda il lato 
pratico, queste tecniche sono state studiante mediante la ricerca di articoli e videotutorial online
sull'applicazione di esse e poi, in una fase successiva, messe in pratica con piccoli esempi 
mediante l'utilizzo di dataset di vario genere, forniti in maniera gratuita da siti web disponibili 
su internet, per poterne capire meglio il funzionamento.\\
Per quanto riguarda il lato teorico di esse è stato necessario studiare una piccola base di statistica
inferenziale ed altre nozioni generali per interpretare al meglio i risultati e le tecniche utilizzate in 
ambito applicativo.\\
\\
Nella seconda fase essendo che i dati forniti erano dati ``grezzi'', prima di passare concretamente 
alla ricerca ed implementazione di un metodo volto a risolvere il problema posto, è stato 
necessario applicare in partenza tutte le tecniche di manipolazione dei dati come filtraggio, 
rinomina delle colonne, tecniche per la sostituzione di valori nulli etc \dots per poter 
ottenere un dataset pulito e lavorabile dal punto di vista applicativo.
