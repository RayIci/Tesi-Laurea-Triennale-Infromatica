\section{Introduzione}
In tale capitolo introduttivo verranno spiegati gli obbiettivi e la suddivisione 
del tirocinio in merito al tempo disponibile, cercando di spiegare, in maniera sintetica, 
come è stato svolto ed organizzato.

\subsection{Premessa}
Lo scopo di questo elaborato è quello di descrivere l'esperienza di tirocinio svoltasi presso 
l'Università Degli Studi di Genova con durata complessiva di 300 ore con inizio 25 novembre 2022
e fine 27 febbraio 2023. Il tirocinio è stato svolto per la maggior parte del tempo da remoto
con incontri, volti all'andamento di esso, sia mediante l'utilizzo di Teams 
(piattaforma sviluppata da Microsoft) che in presenza, presso il dipartimento. 

\subsection{Obbiettivi}
L'obbiettivo di questo tirocinio è stato quello di provare a individuare, se esistenti, uno o più metodi, 
relativi all'analisi di serie temporali, che permettano l'analisi di una serie di 
dati relative a soggetti sani e patologici con lo scopo di analizzare il cammino.

I dati utilizzati sono stati analizzati precedentemente da un gruppo di ricerca del dipartimento
mediante l'ausilio di tecniche differenti da quelle utilizzate durante il tirocinio.
Come risultato finale, in caso di un metodo sicuro e soddisfacente all'analisi, i risultati ottenuti sarebbero
serviti al gruppo di ricerca come un'ulteriore conferma delle analisi da loro eseguite e/o come 
un'analisi da un'ottica differente, da quella adottata da loro, possa comunque portare a conclusioni
simili.

\subsection{Tecnologie utilizzate}
\begin{sloppypar}
Come scelta tecnologica principale, per lo sviluppo dell'intero svolgimento del tirocinio, si è optato
il linguaggio di
programmazione python, dovuto alla sua semplicità, velocità (nella scrittura di codice) e
ampia community, che fornisce molti dei pacchetti utilizzati per eseguire analisi di ogni genere.
Per tenere traccia del lavoro, e per esplorare velocemente le modifiche eseguite, è stato utilizzato
Git come VCS (version control system) e Github come servizio di hosting per i repository.
\end{sloppypar}


\subsection{Suddivisione del lavoro}
Per poter garantire la conoscenza necessaria allo sviluppo di un metodo relativo allo scopo
del tirocinio, il lavoro è stato principalmente suddiviso in due fasi: studio dei metodi
relativi all'analisi di serie temporali e ricerca di un metodo per l'analisi
del problema, con successivo sviluppo.

Nella prima fase sono stati studiati i metodi ed applicazioni di tecniche volte allo studio
di serie temporali, sia da un lato pratico che da un lato teorico. Per quanto riguarda il lato 
pratico, queste tecniche sono state studiante mediante la ricerca di articoli e videotutorial online
sull'applicazione di esse e poi, in una fase successiva, messe in pratica con piccoli esempi 
mediante l'utilizzo di dataset di vario genere, forniti in maniera gratuita da siti web trovati 
su internet, per poterne capire meglio il funzionamento.
Per quanto riguarda il lato teorico di esse è stato necessario studiare una piccola base di statistica
inferenziale ed altre nozioni generali, per interpretare al meglio i risultati e le tecniche utilizzate in 
ambito applicativo.

Nella seconda fase si è passati alla ricerca di un metodo, che utilizzi 
tecniche dell'analisi di serie temporali, per risolvere il problema richiesto.
Prima di essere passare allo sviluppo vero e proprio, 
essendo che i dati forniti erano in uno stato ``grezzo'', è stato 
necessario applicare tutte le tecniche di manipolazione dei dati come filtraggio, 
rinomina delle colonne, tecniche per la sostituzione di valori nulli etc \dots per poter 
ottenere un dataset pulito e lavorabile dal punto di vista applicativo. 