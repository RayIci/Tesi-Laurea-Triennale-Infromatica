\section{Introduzione}
In questo capitolo introduttivo verranno spiegati gli obbiettivi e la suddivisione del tirocinio
in merito al tempo disponibile, cercando di spiegare, in maniera sintetica, come è stato
svolto ed organizzato.

\subsection{Premessa}
Lo scopo di questo elaborato è quello di descrivere l'esperienza di tirocinio svoltasi presso
l'Università Degli Studi di Genova con durata complessiva di circa 300 ore con inizio 25 novembre 2022
e fine 27 febbraio 2023. Il tirocinio è stato svolto per la maggior parte del tempo da remoto
con incontri, volti all'andamento di esso, mediante l'utilizzo di Teams (piattaforma sviluppata da Microsoft)
che in presenza presso il dipartimento. 

\subsection{Obbiettivi}
L'obbiettivo di questo tirocinio è stato quello di provare a trovare, se esistenti, uno o più metodi, 
relativi all'analisi di serie temporali, che permettano l'analisi di una serie di 
dati relative a soggeti sani e petologici con lo scopo di analizzare il cammino.
\\
I dati utilizzati sono stati analizzati precedentemente da un gruppo di ricerca del dipartimento
con tecniche differenti da quelle utilizzate durante il tirocinio
quindi in caso di un metodo sicuro e soddisfacente all'analisi, i risultati ottenuti sarebbero
serviti al gruppo di ricerca come un'ulteriore conferma delle analisi da loro eseguite.

\subsection{Suddivisione del lavoro}
Per poter garantire la conoscienza necessara allo sviluppo di un metodo relativo allo scopo
del tirocinio, il lavoro è stato principalmente suddiviso in due fasi: studio dei metodi
relativi all'analisi di serie temporali ed effettivo sviluppo, e ricerca, di un metodo per l'analisi
del problema posto.